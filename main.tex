%%%%%%%%%%%%%%%%%
% This is an example CV created using altacv.cls (v1.1.5, 1 December 2018) written by
% LianTze Lim (liantze@gmail.com), based on the
% Cv created by BusinessInsider at http://www.businessinsider.my/a-sample-resume-for-marissa-mayer-2016-7/?r=US&IR=T
%%%%%%%%%%%%%%%%

%% If you are using \orcid or academicons
%% icons, make sure you have the academicons
%% option here, and compile with XeLaTeX
%% or LuaLaTeX.
% \documentclass[10pt,a4paper,academicons]{altacv}

%% Use the "normalphoto" option if you want a normal photo instead of cropped to a circle
% \documentclass[10pt,a4paper,normalphoto]{altacv}

\documentclass[10pt,a4paper,ragged2e]{altacv}

%% AltaCV uses the fontawesome and academicon fonts
%% and packages.
%% See texdoc.net/pkg/fontawecome and http://texdoc.net/pkg/academicons for full list of symbols. You MUST compile with XeLaTeX or LuaLaTeX if you want to use academicons.

% Change the page layout if you need to
%% increasing the value of "right" will push content to the left
\geometry{left=1.5cm,right=11cm,marginparwidth=6.8cm,marginparsep=1.2cm,top=1.25cm,bottom=1.25cm}

% Change the font if you want to, depending on whether
% you're using pdflatex or xelatex/lualatex
\ifxetexorluatex
  %% If using xelatex or lualatex:
  \setmainfont{Carlito}
\else
  %% If using pdflatex:
  \usepackage[utf8]{inputenc}
  \usepackage[T1]{fontenc}
  \usepackage[default]{lato}
\fi

\usepackage{hyperref}
\hypersetup{
%   colorlinks   = true,  %%Colours links instead of ugly boxes
  urlcolor      = blue,  %%Colour for external hyperlinks
  linkcolor     = blue,  %%Colour of internal links
  citecolor     = red,   %%Colour of citations
  frenchlinks   = true
}


%% Change the colours if you want to
\definecolor{VividPurple}{HTML}{000000}
\definecolor{SlateGrey}{HTML}{2E2E2E}
\definecolor{LightGrey}{HTML}{2E2E2E}
\colorlet{heading}{VividPurple}
\colorlet{accent}{VividPurple}
\colorlet{emphasis}{SlateGrey}
\colorlet{body}{LightGrey}

%% Change the bullets for itemize and rating marker
%% for \cvskill if you want to
\renewcommand{\itemmarker}{{\small\textbullet}}
\renewcommand{\ratingmarker}{\faCircle}

%% sample.bib contains your publications
\addbibresource{sample.bib}

\begin{document}
\name{Faris Hijazi}
\tagline{Computer Engineer}

%% Profile picture (optional)
% \photo{3.3cm}{profile0.jpg}

\personalinfo{%
  % You can add your own with \printinfo{symbol}{detail}
  \email{\href{mailto:faris@sunwan.com}{faris@sunwan.com}}
  \phone{+966-505501494}
%  \mailaddress{Address, Street, 00000 County}
  \location{Riyadh, Saudi Arabia}
%  \homepage{}
  \github{\href{https://github.com/FarisHijazi}{github.com/FarisHijazi}}
  \linkedin{\href{https://linkedin.com/in/Faris-Hijazi}{/in/Faris-Hijazi}}
%   \twitter{@buzzition}
%   \orcid{orcid.org/0000-0001-6202-8787}
%% If you want to use this field (and also other academicons symbols), add "academicons" option to \documentclass{altacv}
}

%% Makes the header extend all the way to the right.
\begin{fullwidth}
\makecvheader

Detail-oriented Computer Engineer with a background in data science and machine learning.\\
Looking for a job opportunity to tackle real world problems by applying my skills collaborating with teams and developing advanced technical solutions.
I love challenging tasks and learning new technologies and techniques.
I enjoy streamlining processes and automating tedious tasks, like data collection by web-scraping or auto-generating reports/documentation using Python.

\end{fullwidth}

%% Makes fonts of itemize environments slightly smaller
\AtBeginEnvironment{itemize}{\small}

%% Provide the file name containing the sidebar contents as an optional parameter to \cvsection.
%% You can always just use \marginpar{...} if you do
%% not need to align the top of the contents to any
%% \cvsection title in the "main" bar.

\cvsection[page1sidebar]{}
\cvsection{Experience}

\cvevent{Roboticist in Computer Vision (intern)}{National Center for Robotics Technology and Autonomous Systems - King~Abdulaziz~City~for~Science~and~Technology (KACST)}{June~-~August~2019}{Riyadh,Saudi~Arabia}
\verbose{
      Developed \textbf{computer vision algorithms}, submitted a \textbf{research paper} about computer vision, delivered multiple \textbf{project workshops} to KACST faculty and center director.
}{
      \begin{itemize}
            \item Implemented \textbf{computer vision algorithms:} object tracking, image binarization, segmentation.
            \item Submitted a \textbf{research paper} about computer vision.
            \item Prepared and delivered multiple \textbf{project workshops} to KACST faculty and center director.
                  % \item Filed a patent for pest control using robotics.
            \item Worked in an interdisciplinary project and communicated technical ideas with experts in different fields.
                  % \item Read research papers, performed experiments, tested and optimized algorithms.
      \end{itemize}
}

\divider

\cvevent{Teaching Assistant (Part-time)}{KFUPM~College~of~Computer~Science~\&~Engineering}{2017}{Dhahran,Saudi~Arabia}

\divider

\cvevent{Desk Receptionist (Summer job)}{National center of allergy, Asthma and immunology (\href{http://allergyarabia.com}{allergyarabia.com})}{Summer~of~2013}{Riyadh,Saudi~Arabia}
% Organized documents, booked \& checked-in appointments, and welcomed patients.



%\divider

\clearpage
\cvsection[page2sidebar.tex]{}
% %
% % here we're starting the next sidebar (refs section)
% \cvsection[page2sidebar.tex]{}
% \clearpage \cvsection[sections/referees.tex]{}
% %

\cvsection{Projects}

% X Capstone
\cvproject{Capstone: Digitized~attendance~system using face recognition}{
% A service for companies and universities to take attendance using facial recognition built on cloud and edge computing. \small{[Group~project]}\\
}{}
\smallskip

% X KACST
\cvproject{Object tracking using blob detection}{
% Built an object tracking algorithm from scratch. \small{[Group~project~at~KACST]}\\
}{}
\smallskip

\cvproject{Unity Network Simulator (Packet Tracer)}{
A graphical simulator for computer network traffic flow. \small{[Solo~project]}\\
}{FarisHijazi/UnityNetworkPacketTracer}
\smallskip

\cvproject{Differntially-Private KMeans clustering}{
Implemented a differentially-private KMeans, and visualized high-dimensional data (8D) \small{[Solo~project]}\\
}{FarisHijazi/PrivacyEnhancingTechnologies-projects}
\smallskip

\cvproject{Google Images enhancer (browser add-on)}{
\verbose{A browser add-on that enhances the Google.com images page. Brings back ``view~image" button, adds ``download" button, and allows downloading all the full-resolution images on that page. \small{[Solo~project]}\\
}}{FarisHijazi/SuperGoogle}
\smallskip

\cvproject{Verilog LZ4 Decompressor Hardware Implementation}{
\verbose{Designed and synthesized an LZ4 decompressor in FPGA hardware using Verilog HDL. \small{[Course~team~project]}\\
}}{FarisHijazi/LZ4-Decompressor-Verilog}
\smallskip

\cvproject{CPU design - single core 5-stage pipeline}{
\verbose{(Course team project) Designed and simulated a functional pipelined CPU (on Logisim software) that can execute instructions. \small{[Course~team~project]}\\}
}{}
\smallskip

\cvproject{Website enhancer (auto-CAPTCHA) (browser add-on)}{
\verbose{A browser add-on that adds features to a website (rarbg.to) such as auto-CAPTCHA and auto-appending pages via \textit{ajax} calls. \small{[Solo~project]}\\
}}{FarisHijazi/Rarbg-Enhancer-UserScript}
\smallskip



% \clearpage

%% new page here
% 
\cvsection{Education / Courses}

%\divider
\cvevent{B.Sc. in Computer Engineering}{King Fahd University of Petroleum and Minerals (KFUPM)}{August 2015 - June 2020 (expected)}{}
\smallskip
Area-GPA: \textbf{3.5/4} \ CGPA: \textbf{3.3/4} \\
\smallskip
\begin{small}
Key courses: \cvtag{Machine Learning}, \cvtag{Artificial Intelligence}, \cvtag{Design and Synthesis of Digital Systems}, \cvtag{Data Structures}, \cvtag{Computer Networks}
\end{small}

\smallskip
\begin{itemize}
    \item \textbf{Leader of the \href{https://www.robocup.org/}{RoboCup.org} team (embedded~systems)}: Guided the team to constructing a moving robot with wireless communication within a single semester.
    \item Represented the college of computer science and engineering at the \href{https://www.scitech.sa}{SciTech.sa} Expo.
    \item Participated in the KFUPM volunteer day 2016, entertained underprivileged children.
    \item KFUPM Toastmasters club: prepared rooms and maintained club property.
\end{itemize}

% \divider

% \cvevent{High school degree}{Al-Ghad School High School}{2011- 2014}{Riyadh,KSA}
% \smallskip
% Graduated with honors ``Excellent" grade: 98\%

\divider

\cvevent{Deep Learning with PyTorch - NanoDegree}{Udacity}{ Sep 2019 -- Dec 2019}{}
\href{https://classroom.udacity.com/courses/ud188}{https://classroom.udacity.com/courses/ud188}

% \cvevent{Deep Learning Specialization}{Coursera}{ June 2017 -- Aug 2017}{}

%% divider

% 
\cvsection{A Day of My Life}

%% Adapted from @Jake's answer from http://tex.stackexchange.com/a/82729/226
%% Some ad-hoc tweaking to adjust the labels so that they don't overlap
\hspace*{-2cm}
\wheelchart{1.5cm}{0.5cm}{%
  10/10em/accent!30/Sleeping \& dreaming about my projects,
  25/9em/accent!60/Official work,
  % 5/13em/accent!10/\footnotesize\\[1ex],
  20/15em/accent!40/Family time,
  20/8em/accent!20/Researching |\\ personal projects,
  10/8em/accent!20/Working out,
  5/9em/accent/Planning the next day
}

% \clearpage %% this makes a new page like a page break
% \cvsection{References}
\nocite{*}

\printbibliography[heading=pubtype,title={\printinfo{\faBook}{Books}},type=book]

\divider

\printbibliography[heading=pubtype,title={\printinfo{\faFileTextO}{Journal Articles}}, type=article]

\divider

\printbibliography[heading=pubtype,title={\printinfo{\faGroup}{Conference Proceedings}},type=inproceedings]

% *References and copies of certificates will be provided upon request.




%% If the NEXT page doesn't start with a \cvsection but you'd
%% still like to add a sidebar, then use this command on THIS
%% page to add it. The optional argument lets you pull up the
%% sidebar a bit so that it looks aligned with the top of the
%% main column.
% \addnextpagesidebar[-1ex]{page3sidebar}

\mbox{}
\vfill

*References and copies of certificates will be provided upon request.

\end{document}
